
\section{Timeline for Final Project of MUMT620}
	\subsection{10/31/2017 - 11/6/2017}
		\begin{itemize}
			\item I was working on a prototype of the proposal I made
			\item Finished the TouchOSC GUI for it and also the supercollider program
			\item I tested the game and tried to get in the position of an average player
			\item Some things came up immediately, remembering the order of the tones was really difficult because there was no context
			\item This turned out to be a very difficult "ear training" game, with a very slow gameplay
			\item I played some other ear training games to compare, and I realized the experience was much easier in available games
			\item I used this feedback from my own experience to pursue a new idea, taking into account two major points
			\begin{itemize}
				\item The interaction with the controller should be faster and engaging
				\item It should be easy to play the game, even if not doing it "virtuosic", the game mechanics should allow the player to input commands almost without any instructions
			\end{itemize}
			\item During the weekend of November 4th, I came with another idea for a musical controller which could be used for games
			\item The basic controller consists of six discrete commands, inspired by "beat em up" games
			\item The six commands represent three harmonic functions in two modes:
				\begin{itemize}
					\item Tonic, Subdominant, Dominant in a major key
					\item Tonic, Subdominant, Dominant in a relative minor key of the major key above
				\end{itemize}
			\item I finished a basic definition and concept on this week, and the next week I started with the implementation of the controller and game mechanics
		\end{itemize}  

	\subsection{11/7/2017 - 11/13/2017}
		\begin{itemize}
			\item Game mechanics
			\begin{itemize}
				\item The first idea was to have a "turn-based" system in which two players have the opportunity to fight each other by using one of the six commands
				\item The players do not need to be aware of the musical abstraction, however, the transitions are based on harmonic progressions of classical music
				\item The model for the turn based approach was inspired in the "Tick-Tack-Toe" game, with sequences of 3 successive symbols summing a score to a player, the 3 successive symbols are defined by the chord transitions graph
				\item I was researching for previous efforts of a music-based "tick-tack-toe" and found a couple of references
				\begin{itemize}
					\item One of the approaches was an analysis of the game to extract music composition rules
					\item The other one is a "children-friendly" version of tick-tack-toe, for teaching music concepts
				\end{itemize}
				\item By the end of this week, I do not have a fully-defined mechanic, but a prototype of a transition graph and game mechanics
			\end{itemize}
			\item Game controller
			\begin{itemize}
				\item During this week a basic version of the controller was constructed
				\item The basic guidelines for the design of the controller were the following:
				\begin{itemize}
					\item The controller should turn a smartphone into a controller, i.e. take advantage of gestures that can be performed in a smartphone
					\item The six basic commands from the controller are captured from rotating the device, these commands are discrete gestures, however, a normalized value of the gyroscope is used stored in a variable and can be used for controlling other parameters not related to the basic six commands
					\item The sensor used as a reference is the embedded Bosch Sensortec BMI160 accelerometer and gyroscope in Nexus 6p smartphones (my personal smartphone)
					\item The simplest version of the controller just outputs six different events with six different rotation gestures
				\end{itemize}
				\item By the end of this week, the simplest version of the controller is tested and working in a Nexus 6p smartphone, with a very basic graphic interface to visualize the events triggered by the gestures and sample sounds synthesized
			\end{itemize}
		\end{itemize}
		\subsection{11/14/2017 - 11/20/2017}
			\begin{itemize}
				\item The work for this week is divided in two parts:
				\item Game mechanics
				\begin{itemize}
					\item Finish a simple game mechanic concept and state machine that uses only the basic gestures of the controller
				\end{itemize}
				\item Game controller
				\begin{itemize}
					\item Incorporate continuous parameters into the controller and design a user interface that uses these new parameters
				\end{itemize}
			\end{itemize}
		\subsection{11/21/2017 - 27/20/2017}
			\begin{itemize}
				\item The final project will consist as well of two parts
				\begin{itemize}
					\item A game controller with an interface that makes use of all its parameters, without any game mechanics implementation
					\item A simple game mechanic that uses the basic functionality of the game controller
				\end{itemize}
			\end{itemize}
